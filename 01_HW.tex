% 01_HW.tex
%------------------------------------------------------------
%   DOCUMENT AND PACKAGES CONFIGURATIONS
%-----------------------------------------------------------
\documentclass[12pt]{article}
\usepackage{etex}
\usepackage{textcomp}
\usepackage{gensymb}
\usepackage{exsheets}                   % Question and solution environments
\usepackage{amsmath, amssymb, amsthm}   % Standard AMS packages
\usepackage{esint}                      % Integral signs
\usepackage{gensymb}                    % Miscellaneous symbols
\usepackage{commath}                    % Differential symbols
\usepackage{xcolor}                     % Colours
\usepackage{cancel}                     % Cancelling terms
\usepackage{amsfonts}					% Math fonts
\usepackage{nicefrac} 				   	% Slanted fractions
\usepackage[hmargin=1.5cm,vmargin=1.5cm]{geometry}
\usepackage[super,comma,sort&compress, square]{natbib}
\usepackage{graphicx}					% Inset Graphics
\usepackage{tikz}						% Drawing 
\usepackage{pgfplots}
\usepackage{float}
\usepackage{csvsimple}
\usepackage{gnuplottex}
\usepackage{todonotes}
\usepackage{csquotes}
\usepackage{pgfkeys}

\pgfplotsset{compat=1.7}
\usepackage[english]{babel}
\pgfplotsset{every axis y label/.append style={xshift=1.5em}}

\usepackage{hyperref} % to turn the links clickable and load this before cleveref
\usepackage{cleveref}

\crefname{table}{table}{tables}
\Crefname{table}{Table}{Tables}
\crefname{figure}{figure}{figures}
\Crefname{figure}{Figure}{Figures}
\crefname{equation}{equation}{equations}
\Crefname{Equation}{Equation}{Equation}

%---------------------------------------------------------------------------
% 							TITLE PAGE
%---------------------------------------------------------------------------

\title{Electromagnetic Fields }
\author{Scott Skrobel}
\date{Feburary 2, 2019}

\begin{document}
\maketitle

\section{Plane Wave Solution}
	Consider the electric field specified in time, and space by the function
	\( \vec{E}(z,t) = \vec{a_(x)} E_0 cos(\omega t - \beta z) \) where 
	\( E_{0}, \omega, \) $\beta$ are constants. Using only the differential
	form of Maxwell’s equations find the magnetic field that must coexist
	with the given electric field in free space.
	\vspace{5mm}

	Maxwell's equations in differential form.
	\begin{flalign}
		& &\nabla \cdot \mathbf{D}& = \rho_{v} &&\text{\llap{(Gauss)}} \\[0.5ex]
		& &\nabla \cdot \mathbf{B}& = 0 &&\text{\llap{(Gauss)}} \\[0.5ex]
		& &\nabla \times \mathbf{E}& = -\frac{\partial \mathbf{B}}{\partial t} &&\text{\llap{(Faraday)}} \\[0.5ex]
		& &\nabla \times \mathbf{B}& = \mu(\mathbf{J} + \varepsilon_0 \frac{\partial \mathbf{E}} {\partial t}) &&\text{\llap{(Ampere)}}
	\end{flalign}

	By Faraday's Law, The spatial variation of the electric field gives rise to a time-varying Magnetic 
	field. By taking the curl of Amepere  we arrive at the result seen in \cref{eq:curl} 
	\begin{equation}
		\label{eq:curl}
		\nabla\times (\nabla \times \mathbf{B}) =\nabla \times \mu\left( \mathbf{J} +\epsilon \frac{\partial \mathbf{E}} {\partial t} \right)=
		\mu \nabla \times  \mathbf{J} +\mu \epsilon \nabla\times \frac{\partial \mathbf{E}} {\partial t}
	\end{equation}
	
	Using the current Density in \cref{eq:ohm} we can substitute into \cref{eq:curl} to get \cref{eq:curlJ}
    \begin{equation}
		\label{eq:ohm}
		\mathbf{J}=\sigma\mathbf{E}
	\end{equation}
	\noindent
    Substitution of \cref{eq:ohm} into \cref{eq:curl} yields 

	\begin{equation}
		\label{eq:curlJ}
		 \nabla \times(\nabla \times \mathbf{B}) =\mu \sigma\nabla \times \mathbf{E}+\mu \epsilon \nabla \times  \frac{\partial \mathbf{E}} {\partial t} 
	\end{equation}
	\noindent

	The spatial variation of the electric field gives rise to a time-varying Magnetic field, Shown in \cref{eq:Farady},
	allows us to perform another substitution, into \cref{eq:curlJ} which yields an expression for the RHS as is shown in 
	\cref{eq:rhs}

	\begin{equation}
		\label{eq:Farady}
		\nabla \times \mathbf{E} = - \frac{\partial \mathbf{B}} {\partial t}
	\end{equation}

	\begin{equation}
		\label{eq:rhs}
		\nabla \times(\nabla \times \mathbf{B}) = -\mu\epsilon\frac{\partial^{2} \mathbf{B}} {\partial t^{2}} - \mu\sigma\frac{\partial \mathbf{B}} {\partial t}
	\end{equation}

	The vector identity from \cref{eq:vect2} can be used to get the result in \cref{eq:lhs2}.

	\begin{equation}
		\label{eq:vect2}
		\nabla \times \left( \nabla \times \mathbf{\psi} \right) = \nabla(\nabla \cdot \mathbf{\psi}) - \nabla^{2}\mathbf{\psi}
    \end{equation}
	
	\begin{equation}
		\label{eq:lhs}
		\nabla \times(\nabla \times \mathbf{B} =(\nabla \times (\nabla \times \mathbf{B} ))=\nabla(\nabla\cdot\mathbf{B})-\nabla^{2}\mathbf{B}
	\end{equation}
		
	\noindent
	Where $\nabla(\nabla\cdot\mathbf{B})=0$ \cref{eq:lhs} becomes \cref{eq:rhs}
		
	\begin{equation}
		\label{eq:lhs2}
		\nabla \times(\nabla \times \mathbf{B}) =\nabla^{2}\mathbf{B}
	\end{equation}
		
	\noindent
	The last step is to simply equate \cref{eq:rhs} by  \cref{eq:lhs} as is shown in \cref{eq:penultimate}, then multiply the whole thing through by $-1$ to get the same expression as is in 
		
	\begin{equation}
		\label{eq:penultimate}
		\nabla \times(\nabla \times \mathbf{B}) = -\nabla^{2}\mathbf{B} = -\epsilon\frac{\partial^{2}\mathbf{B}} {\partial t^{2}} - \sigma\frac{\partial \mathbf{B}} {\partial t}= [-\mu\epsilon\frac{\partial^{2} \mathbf{B}} {\partial t^{2}} - \mu\sigma\frac{\partial \mathbf{B}} {\partial t}]
	\end{equation}

	\begin{equation}
		\label{eq:identity}
		\nabla^{2}\mathbf{B}=\epsilon\mu \frac{\partial^{2} \mathbf{B}}{\partial t^{2}}+\sigma\mu \frac{\partial \mathbf{B}}{\partial t}
	\end{equation}

	The simplest solutions to the differential equations are sinusoidal wave functions:
	\begin{equation}
		\label{eq:Efield}
		\vec{E}(z,t) = \vec{a_(x)} E_{0} cos(\omega t - \beta z)
	\end{equation}
 
	\begin{equation}
		\label{eq:Bfield}
		\vec{B}(z,t) =  \vec{a_(y)}B_{0} cos(\omega t - \beta z)
	\end{equation}

	where \( \beta = \frac{2\pi}{\lambda} \) 
	is the the wavenumber, 
	\( \omega = 2\pi f \) is the angular frequency,
	\( \lambda \) is the wavelength, f is the
	frequency and \() \frac{\omega}{\beta} = \lambda f = c \) = speed of light \\

	Taking the partial derivatives of E and B and substitute into \cref{eq:Farady}

	\begin{equation}
		\frac{\partial \mathbf{E}} {\partial t} = \beta E_{0} sin(\omega t - \beta z)	
	\end{equation}

	\begin{equation}
		\frac{\partial \mathbf{B}} {\partial t} = \omega B_{0} sin(\omega t - \beta z)
	\end{equation}

	\begin{equation}
		\beta E_{0} sin(\omega t - \beta z)	= \omega E_{0} sin(\omega t - \beta z)
	\end{equation}

	\begin{equation}
		E_{0} = \frac{\omega}{\beta} B_{0} = cB_{0}		
	\end{equation}

	Giving Rise to the ratio of the electric and magnetic field amplitudes at every instant equations the speed of light.

	\begin{equation}
		\frac{E}{B} = c
	\end{equation}


























\end{document}